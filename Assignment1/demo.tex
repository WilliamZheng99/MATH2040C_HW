\documentclass[12pt]{article}%
\usepackage{amsfonts}
\usepackage{fancyhdr}
\usepackage{comment}
\usepackage[a4paper, top=2.5cm, bottom=2.5cm, left=2.2cm, right=2.2cm]%
{geometry}
\usepackage{times}
\usepackage{amsmath}
\usepackage{changepage}
\usepackage{stfloats}
\usepackage{amssymb}
\usepackage{graphicx}
\usepackage{indentfirst}
\setlength{\parindent}{2em}
\setcounter{MaxMatrixCols}{30}
\newtheorem{theorem}{Theorem}
\newtheorem{acknowledgement}[theorem]{Acknowledgement}
\newtheorem{algorithm}[theorem]{Algorithm}
\newtheorem{axiom}{Axiom}
\newtheorem{case}[theorem]{Case}
\newtheorem{claim}[theorem]{Claim}
\newtheorem{conclusion}[theorem]{Conclusion}
\newtheorem{condition}[theorem]{Condition}
\newtheorem{conjecture}[theorem]{Conjecture}
\newtheorem{corollary}[theorem]{Corollary}
\newtheorem{criterion}[theorem]{Criterion}
\newtheorem{definition}[theorem]{Definition}
\newtheorem{example}[theorem]{Example}
\newtheorem{exercise}[theorem]{Exercise}
\newtheorem{lemma}[theorem]{Lemma}
\newtheorem{notation}[theorem]{Notation}
\newtheorem{problem}[theorem]{Problem}
\newtheorem{proposition}[theorem]{Proposition}
\newtheorem{remark}[theorem]{Remark}
\newtheorem{solution}[theorem]{Solution}
\newtheorem{summary}[theorem]{Summary}
\newenvironment{proof}[1][Proof]{\textbf{#1.} }{\ \rule{0.5em}{0.5em}}

\usepackage{mathtools}

\newcommand{\Q}{\mathbb{Q}}
\newcommand{\R}{\mathbb{R}}
\newcommand{\C}{\mathbb{C}}
\newcommand{\Z}{\mathbb{Z}}

\begin{document}

\title{MATH2040C Homework 1}
\author{Zheng Weijia (William)}
\date{\today}
\maketitle

\section{Section 1.2, Q13}

To check if a set is a vector space, one need to check those VS's. 

~\

[VS1]: $\forall (a_1,a_2),(b_1,b_2) \in \mathbb{V}$, note that from definition, $$(a_1,a_2)+(b_1,b_2)=(a_1+b_1,a_2b_2)$$ 

and $$(b_1,b_2)+(a_1,a_2)=(a_1+b_1,a_2b_2)$$

Hence $(b_1,b_2)+(a_1,a_2) = (a_1,a_2)+(b_1,b_2), \forall (a_1,a_2),(b_1,b_2) \in \mathbb{V}.$ Therefore VS1 is satisfied.

~\

[VS2]: $\forall (a_1,a_2),(b_1,b_2),(c_1,c_2) \in \mathbb{V}$, note that by definition, $$((a_1,a_2)+(b_1,b_2))+(c_1,c_2)=(a_1+b_1,a_2b_2)+(c_1,c_2)=(a_1+b_1+c_1,a_2b_2c_2)$$

and $$(a_1,a_2)+((b_1,b_2)+(c_1,c_2)) = (a_1,a_2)+(b_1+c_1,b_2c_2)=(a_1+b_1+c_1,a_2b_2c_2)$$ $$\therefore (a_1,a_2)+((b_1,b_2)+(c_1,c_2)) = ((a_1,a_2)+(b_1,b_2))+(c_1,c_2), \forall (a_1,a_2),(b_1,b_2),(c_1,c_2) \in \mathbb{V}.$$ Therefore, VS2 is satisfied.

~\

[VS3]: Note that an element $(0,1)\in \mathbb{V}.$ Note that $\forall (a_1,a_2)\in \mathbb{V},$ $$(0,1)+(a_1,a_2)=(0+a_1,1\cdot a_2) = (a_1,a_2).$$

Hence VS3 is satisfied. 

~\

[VS4]: Note that $(1,0)\in \mathbb{V}.$ 

And $\forall (a_1,a_2)\in \mathbb{V}, (1,0)+(a_1,a_2)=(1+a_1,0)\neq (0,1).$ Note that the $(0,1)$ is the zero vector we defined in order to satisfy VS3.

Therefore VS4 cannot be satisfied, hence $\mathbb{V}$ is not a vector space under the operations stated in the question.

~\

\section{Section 1.2 Q21}

To check if a set is a vector space, one need to check those VS's.

[VS1]: $\forall (v_1,w_1),(v_2,w_2)\in Z $, note that $$(v_1,w_1)+(v_2,w_2)=(v_1+v_2,w_1+w_2)=(v_2,w_2)+(v_1,w_1).$$ 

Therefore, VS1 is satisfied. 

~\

[VS2]: $\forall (v_1,w_1),(v_2,w_2),(v_3,w_3)\in Z$, note that $$((v_1,w_1)+(v_2,w_2))+(v_3,w_3)=(v_1+v_2,w_1+w_2)+(v_3,w_3)=(v_1+v_2+v_3,w_1+w_2+w_3).$$ And $$(v_1,w_1)+((v_2,w_2)+(v_3,w_3))=(v_1,w_1)+(v_2+v_3,w_2+w_3)=(v_1+v_2+v_3,w_1+w_2+w_3)$$

Therefore $((v_1,w_1)+(v_2,w_2))+(v_3,w_3)=(v_1,w_1)+((v_2,w_2)+(v_3,w_3))$, which implies that VS2 is satisfied. 

~\

[VS3]: Denote $0_V$ is a zero vector of $V$ and $0_W$ is a zero vector of $W$. 

Note that $(0_V,0_W) \in Z.$

And $\forall (v,w)\in Z,$ $$(0_V,0_W)+(v,w) = (0_V+v,0_W+w)=(v,w).$$ 

Therefore, VS3 is satisfied, and we also define $0_Z = (0_V,0_W)$ as a zero vector of $Z$.

~\

[VS4]: $\forall (v,w) \in Z,$ note that $\exists \hat v \in V, \hat w \in W$ such that $v+\hat v=0_V, w+\hat w = 0_W$ because $V$ and $W$ are themselves vector spaces.

Note that $(\hat v, \hat w) \in Z$, since $\hat v \in V, \hat w \in W$ and $$(v,w)+(\hat v, \hat w)=(v+\hat v, w+\hat w)=(0_V,0_W)=0_Z.$$

Therefore, VS4 is satisfied. 

~\

[VS5]: Note that $1\in \mathbb{F}$ and $\forall (v,w)\in Z,$ $$ 1\cdot (v,w)=(1\cdot v, 1\cdot w)=(v,w).$$

Therefore, VS5 is satisfied. 

~\ 

[VS6]: Note that $\forall (v,w)\in Z, \forall a,b \in \mathbb{F},$ $$(ab)(v,w)=(ab\cdot v,ab\cdot w)=(a)(b\cdot v,b\cdot w)=a(b(v,w)).$$

Therefore, VS6 is satisfied. 

~\ 

[VS7]: Note that $\forall (v_1,w_1), (v_2,w_2)\in Z, \forall a \in \mathbb{F},$ $$a((v_1,w_1)+(v_2,w_2))=a(v_1+v_2,w_1+w_2)=(a\cdot v_1 + a\cdot v_2, a\cdot w_1 + a\cdot w_2)=a(v_1,w_1)+a(v_2,w_2).$$ 

Note that the second equailty holds for $V$ and $W$ themselves being vector spaces and $v_1,v_2 \in V, w_1, w_2\in W.$

Therefore, VS7 is satisfied. 

~\ 

[VS8]: Note that $\forall (v,w)\in Z, \forall a,b \in \mathbb{F},$ $$(a+b)(v,w)=((a+b)\cdot v,(a+b)\cdot w )$$ 

Note that $V,W$ are vector spaces over field $\mathbb{F}$, therefore $$(a+b)v=a\cdot v + b\cdot v,$$ $$(a+b)w=a\cdot w + b\cdot w.$$

Hence $$(a+b)(v,w)=(a\cdot v + b\cdot v,a\cdot w + b\cdot w)=(a\cdot v, a\cdot w)+ (b\cdot v, b\cdot w)=a(v,w)+b(v,w).$$

Therefore, VS8 is satisfied. 

~\ 

Since the requirements are all satisfied, therefore the set $Z$ is a vector space over $\mathbb{F}$ with the operations stated in the question.

~\ 

\section{Section 1.3 Q11}



\end{document}
