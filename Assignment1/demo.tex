\documentclass[12pt]{article}%
\usepackage{amsfonts}
\usepackage{fancyhdr}
\usepackage{comment}
\usepackage[a4paper, top=2.5cm, bottom=2.5cm, left=2.2cm, right=2.2cm]%
{geometry}
\usepackage{times}
\usepackage{amsmath}
\usepackage{changepage}
\usepackage{stfloats}
\usepackage{amssymb}
\usepackage{graphicx}
\usepackage{indentfirst}
\setlength{\parindent}{2em}
\setcounter{MaxMatrixCols}{30}
\newtheorem{theorem}{Theorem}
\newtheorem{acknowledgement}[theorem]{Acknowledgement}
\newtheorem{algorithm}[theorem]{Algorithm}
\newtheorem{axiom}{Axiom}
\newtheorem{case}[theorem]{Case}
\newtheorem{claim}[theorem]{Claim}
\newtheorem{conclusion}[theorem]{Conclusion}
\newtheorem{condition}[theorem]{Condition}
\newtheorem{conjecture}[theorem]{Conjecture}
\newtheorem{corollary}[theorem]{Corollary}
\newtheorem{criterion}[theorem]{Criterion}
\newtheorem{definition}[theorem]{Definition}
\newtheorem{example}[theorem]{Example}
\newtheorem{exercise}[theorem]{Exercise}
\newtheorem{lemma}[theorem]{Lemma}
\newtheorem{notation}[theorem]{Notation}
\newtheorem{problem}[theorem]{Problem}
\newtheorem{proposition}[theorem]{Proposition}
\newtheorem{remark}[theorem]{Remark}
\newtheorem{solution}[theorem]{Solution}
\newtheorem{summary}[theorem]{Summary}
\newenvironment{proof}[1][Proof]{\textbf{#1.} }{\ \rule{0.5em}{0.5em}}

\usepackage{mathtools}

\newcommand{\Q}{\mathbb{Q}}
\newcommand{\R}{\mathbb{R}}
\newcommand{\C}{\mathbb{C}}
\newcommand{\Z}{\mathbb{Z}}

\begin{document}

\title{MATH2040C Homework 1}
\author{ZHENG Weijia (William)}
\date{\today}
\maketitle

\section{Section 1.2, Q13}

To check if a set is a vector space, one need to check those VS's. 

~\

[VS1]: $\forall (a_1,a_2),(b_1,b_2) \in \mathbb{V}$, note that from definition, $$(a_1,a_2)+(b_1,b_2)=(a_1+b_1,a_2b_2)$$ 

and $$(b_1,b_2)+(a_1,a_2)=(a_1+b_1,a_2b_2)$$

Hence $(b_1,b_2)+(a_1,a_2) = (a_1,a_2)+(b_1,b_2), \forall (a_1,a_2),(b_1,b_2) \in \mathbb{V}.$ Therefore VS1 is satisfied.

~\

[VS2]: $\forall (a_1,a_2),(b_1,b_2),(c_1,c_2) \in \mathbb{V}$, note that by definition, $$((a_1,a_2)+(b_1,b_2))+(c_1,c_2)=(a_1+b_1,a_2b_2)+(c_1,c_2)=(a_1+b_1+c_1,a_2b_2c_2)$$

and $$(a_1,a_2)+((b_1,b_2)+(c_1,c_2)) = (a_1,a_2)+(b_1+c_1,b_2c_2)=(a_1+b_1+c_1,a_2b_2c_2)$$ $$\therefore (a_1,a_2)+((b_1,b_2)+(c_1,c_2)) = ((a_1,a_2)+(b_1,b_2))+(c_1,c_2), \forall (a_1,a_2),(b_1,b_2),(c_1,c_2) \in \mathbb{V}.$$ Therefore, VS2 is satisfied.

~\

[VS3]: Note that an element $(0,1)\in \mathbb{V}.$ Note that $\forall (a_1,a_2)\in \mathbb{V},$ $$(0,1)+(a_1,a_2)=(0+a_1,1\cdot a_2) = (a_1,a_2).$$

Hence VS3 is satisfied. 

~\

[VS4]: Note that $(1,0)\in \mathbb{V}.$ 

And $\forall (a_1,a_2)\in \mathbb{V}, (1,0)+(a_1,a_2)=(1+a_1,0)\neq (0,1).$ Note that the $(0,1)$ is the zero vector we defined in order to satisfy VS3.

Therefore VS4 cannot be satisfied, hence $\mathbb{V}$ is not a vector space under the operations stated in the question.

~\

\section{Section 1.2 Q21}

To check if a set is a vector space, one need to check those VS's.

[VS1]: $\forall (v_1,w_1),(v_2,w_2)\in Z $, note that $$(v_1,w_1)+(v_2,w_2)=(v_1+v_2,w_1+w_2)=(v_2,w_2)+(v_1,w_1).$$ 

Therefore, VS1 is satisfied. 

~\

[VS2]: $\forall (v_1,w_1),(v_2,w_2),(v_3,w_3)\in Z$, note that $$((v_1,w_1)+(v_2,w_2))+(v_3,w_3)=(v_1+v_2,w_1+w_2)+(v_3,w_3)=(v_1+v_2+v_3,w_1+w_2+w_3).$$ And $$(v_1,w_1)+((v_2,w_2)+(v_3,w_3))=(v_1,w_1)+(v_2+v_3,w_2+w_3)=(v_1+v_2+v_3,w_1+w_2+w_3)$$

Therefore $((v_1,w_1)+(v_2,w_2))+(v_3,w_3)=(v_1,w_1)+((v_2,w_2)+(v_3,w_3))$, which implies that VS2 is satisfied. 

~\

[VS3]: Denote $0_V$ is a zero vector of $V$ and $0_W$ is a zero vector of $W$. 

Note that $(0_V,0_W) \in Z.$

And $\forall (v,w)\in Z,$ $$(0_V,0_W)+(v,w) = (0_V+v,0_W+w)=(v,w).$$ 

Therefore, VS3 is satisfied, and we also define $0_Z = (0_V,0_W)$ as a zero vector of $Z$.

~\

[VS4]: $\forall (v,w) \in Z,$ note that $\exists \hat v \in V, \hat w \in W$ such that $v+\hat v=0_V, w+\hat w = 0_W$ because $V$ and $W$ are themselves vector spaces.

Note that $(\hat v, \hat w) \in Z$, since $\hat v \in V, \hat w \in W$ and $$(v,w)+(\hat v, \hat w)=(v+\hat v, w+\hat w)=(0_V,0_W)=0_Z.$$

Therefore, VS4 is satisfied. 

~\

[VS5]: Note that $1\in \mathbb{F}$ and $\forall (v,w)\in Z,$ $$ 1\cdot (v,w)=(1\cdot v, 1\cdot w)=(v,w).$$

Therefore, VS5 is satisfied. 

~\ 

[VS6]: Note that $\forall (v,w)\in Z, \forall a,b \in \mathbb{F},$ $$(ab)(v,w)=(ab\cdot v,ab\cdot w)=(a)(b\cdot v,b\cdot w)=a(b(v,w)).$$

Therefore, VS6 is satisfied. 

~\ 

[VS7]: Note that $\forall (v_1,w_1), (v_2,w_2)\in Z, \forall a \in \mathbb{F},$ $$a((v_1,w_1)+(v_2,w_2))=a(v_1+v_2,w_1+w_2)=(a\cdot v_1 + a\cdot v_2, a\cdot w_1 + a\cdot w_2)=a(v_1,w_1)+a(v_2,w_2).$$ 

Note that the second equailty holds for $V$ and $W$ themselves being vector spaces and $v_1,v_2 \in V, w_1, w_2\in W.$

Therefore, VS7 is satisfied. 

~\ 

[VS8]: Note that $\forall (v,w)\in Z, \forall a,b \in \mathbb{F},$ $$(a+b)(v,w)=((a+b)\cdot v,(a+b)\cdot w )$$ 

Note that $V,W$ are vector spaces over field $\mathbb{F}$, therefore $$(a+b)v=a\cdot v + b\cdot v,$$ $$(a+b)w=a\cdot w + b\cdot w.$$

Hence $$(a+b)(v,w)=(a\cdot v + b\cdot v,a\cdot w + b\cdot w)=(a\cdot v, a\cdot w)+ (b\cdot v, b\cdot w)=a(v,w)+b(v,w).$$

Therefore, VS8 is satisfied. 

~\ 

Since the requirements are all satisfied, therefore the set $Z$ is a vector space over $\mathbb{F}$ with the operations stated in the question.

~\ 

\section{Section 1.3 Q11}

$\forall n\geq 1$ and $n$ being an integer, note that $f_1(x) = x^{n}+1 \in W$ and $f_2(x)=-x^{n} \in W.$

Given that $n\geq 1$, suppose that $W$ is a subspace of $P(\mathbb{F}).$ Then $W$ is a vector space itself, which implies that $$f_1(x)+f_2(x)=1\in W.$$

Note that $1$ is of degree 0, and $1\neq 0.$ Hence by definition of $W$, $1=f_1(x)+f_2(x)\notin W.$

This is violating the requirements of being a vector space, because the addition defined on $W$, which is supposed to be a vector space, should have range $W$.

Therefore, W is not a subspace of $P(F)$ at the first place. 

~\ 

\section{Section 1.3 Q19}

First, we prove the "if" direction. 

Given that $W_1\subset W_2$ or $W_2 \subset W_1,$ we would prove $W_1 \cup W_2$ is a subspace of $V$.

Suppse the case is that $W_1\subset W_2,$ then $W_1 \cup W_2= W_2.$ From the condition we knwo that $W_2$ itself is a subspace of $V.$ Therefore $W_1 \cup W_2= W_2$ is a subspace of $V.$

Then, suppse the case is that $W_2\subset W_1,$ then $W_1 \cup W_2= W_1.$ From the condition we knwo that $W_2$ itself is a subspace of $V.$ Therefore $W_1 \cup W_2= W_1$ is a subspace of $V.$

The "if" direction is proved. 

~\ 

We would prove the "only if" part then. Now we assume $W_1 \cup W_2$ is a subspace of $V$ and try to deduce that $W_1 \subset W_2$ or $W_1 \subset W_2.$

Assume it's not the case, neither of $W_1$ and $W_2$ can be empty set. 

Then $\exists x_1 \in W_1$ such that $x_1 \notin W_2,$ and $\exists x_2 \in W_2$ such that $x_2 \notin W_1$. 
Note that $W_1,W_2$ are subspaces of $W_1 \cup W_2$, hence the zero vector of $W_1 \cup W_2$'s (denoted as $0_{12}$) is also $W_1$'s (denoted as $0_1$) and $W_2$'s (denoted as $0_2$). 

In short, $$0_{12}=0_1=0_2.$$

Note that $x_1+x_2 \in W_1 \cup W_2,$ since both $x_1,x_2 \in W_1 \cup W_2.$

(i) Suppose $x_1+x_2 \in W_1.$ As $W_1$ itself is a vector space, $\exists y_1 \in W_1$ such that $x_1+y_1 = 0_1.$ Then $$y_1+x_1+x_2=0_1+x_2=0_2+x_2=x_2 \in W_1.$$

Which contradicts with our assumption that $x_2 \notin W_1.$ 

~\ 

(ii) Suppose $x_1+x_2 \in W_2.$ As $W_2$ itself is a vector space, $\exists y_2 \in W_2$ such that $x_2+y_2 = 0_2.$ Then $$x_1+x_2+y_2=x_1+0_2=x_1+0_1=x_1 \in W_2.$$

Which contradicts with our assumption that $x_1 \notin W_2.$ 

Therefore, $\forall x_1 \in W_1, x_1 \in W_2$ or $\forall x_2 \in W_2, x_2 \in W_1$ must hold at the first place. Which is by definition $W_1 \subset W_2$ or $W_2 \subset W_1.$

~\ 

\section{Section 1.3 Q31}

\subsection{(a)} 

First we prove the "if" part, which assumes $v \in W$

~\ 

$\forall v+x \in v+W$, where $x \in W$, note that $v+x \in W$ since $W$ is a vector space. Hence $v+W \subset W.$

~\ 

$\exists y \in W$ such that $y+v=0_W$, where $0_W$ is the zero vector of $W$. 
$\forall x \in W,$ $$x=x+0_W=x+(y+v)=(x+y)+v.$$ 

Note that $x+y\in W$, hence $x=(x+y)+v \in W.$ Therefore, $W \subset v+W.$ Then $v+W=W$, hence $v+W$ is a subspace.

~\ 

Then we would prove the "only if" part, which assumes $v+W$ is a subspace of $V$. Note that $\forall x \in W, v+x \in v+W.$

Because  $x\in W$ and $W$ is a vector space, $\exists y \in W$ such that $x+y=0_W$, where $0_W$ is the zero vector of $W$. Also note that $v+y \in v+W.$

Then $v+x+(v+y)=v+(x+y+v) \ v+W,$ since $v+W$ is a subspace. Hence $x+y+v \in W.$ Recall that $x+y=0_W$, then $v \in W.$

Therefore, we proved the (a) part. 

~\ 

\subsection{(b)} 

We prove the "if" part first, which assumes $v_1-v_2 \in W$.

Note that $\forall v_1+w_1 \in v_1+W$, since $v_1-v_2 \in W$ as assumed and $w_1 \in W$, we have $$v_1-v_2+w_1 \in W.$$

Therefore, there exists an element in $v_2+W$, which is $v_2+v_1-v_2+w_1=v_1+w_1\in v_2+W.$ 

Hence $v_1+W \subset v_2+W.$

~\ 

Note that $\forall v_2+w_2 \in v_2+W$, since $v_1-v_2 \in W$ as assumed and $w_2 \in W$, we have $$v_1-v_2-w_2 \in W.$$

Therefore, there exists an element in $v_1+W$, which is $v_1-(v_1-v_2-w_2)=v_2+w_2\in v_1+W.$ 

Hence $v_2+W \subset v_1+W.$ Therefore, $v_2+W = v_1+W.$

~\ 

We prove the "only if" part then, which assumes $v_1+W = v_2+W$. 

Note that $\forall v_1+w_1\in v_1+W$, $v_1+w+1 \in v_2 +W.$ Therefore $\exists w_2 \in W$ such that $v_2+w_2=v_1+w_1.$ Which implies $$v_1-v_2=w_2-w_1.$$

Note that $w_2,w_1 \in W$, and $1,-1 \in \mathbb{F},$ then $w_2-w_1=v_1-v_2\in W.$

Therefore we proved the (b) part. 

~\ 

\subsection{(c)} 

From (b) part, since $v_1+W=v_1^{'}+W$ and $v_2+W=v_2^{'}+W$, we have $v_1-v_1^{'}\in W$ and $v_2-v_2^{'} \in W.$ 

Hence $$v_1-v_1^{'}+v_2-v_2^{'}=v_1+v_2-v_1^{'}-v_2^{'}\in W.$$

By definition, $(v_1+W)+(v_2+W)=(v_1+v_2)+W.$ and $(v_1^{'}+W)+(v_2^{'}+W)=(v_1^{'}+v_2^{'})+W.$ Combine with the equation two lines above and result from (b), $$(v_1+W)+(v_2+W)=(v_1^{'}+W)+(v_2^{'}+W).$$

By definition, $a(v_1+W)=av_1+W$, $a(v_1^{'}+W)=av_1^{'}+W.$ With $v_1-v_1^{'} \in W$ and $a\in \mathbb{F},$ $a(v_1-v_1^{'})=av_1-av_1^{'}\in W.$

Therefore $a(v_1+W)=av_1+W=av_1^{'}+W=a(v_1^{'}+W).$

Therefore we proved the (c) part. 

~\ 

\subsection{(d)}

To prove $S$ is a vector space as stated. One need to check the VS's. 

Note that V is a vector space, and W is a subspace of V over field $\mathbb{F}.$ And also note that the operations are well defined that fits the requirements. 

[VS1]: $\forall v_1+W,v_2+W \in S$, $$v_1+W+(v_2+W) =(v_1+v_2)+W=(v_2+v_1)+W=v_2+W+(v_1+W).$$ 

Therefore, VS1 is satisfied. 


~\ 

[VS2]: $\forall v_1+W,v_2+W,v_3+W \in S,$ 

Note that $(v_1+W+(v_2+W))+v_3+W=(v_1+v_2+v_3)+W=v_1+W+(v_2+v_3)+W=v_1+W+(v_2+W+v_3+W).$ 

Therefore, VS2 is satisfied. 

~\ 

[VS3]: Since $V$ is a vector space, denote the zero vector of $V$ as $0_V.$ Note that $0_V +W \in S$ $$\forall v_1+W \in S, v_1+W+(0_V +W)=(v_1+0_V)+W=v_1+W.$$

Hence $0_V + W $ is the zero vector of $S.$ And the VS3 is satisfied. 

~\ 

[VS4]: Since $V$ is a vector space, $\forall v_1 \in V,$ 

$\exists u_1 \in V$ such that $v_1+u_1=0_V.$ For $v_1+W \in S,$ $v_1+W+(u_1+W)=(v_1+u_1)+W=0_V+W.$ Which is the zero vector of $S.$

Therefore, VS4 is satisfied. 

~\ 

[VS5]: Note that $\forall v_1+W\in S,$ $1\cdot (v_1+W)=1\cdot v_1 +W = v_1+W.$

Therefore, VS5 is satisfied. 

~\ 

[VS6]:  Note that $\forall v_1+W\in S, \forall a,b \in \mathbb{F},$ $$(ab)(v_1+W)=ab\cdot v_1+W=a(b\cdot v_1 +W)=a(b(v_1+W)).$$

Therefore, VS6 is satisfied. 

~\ 

[VS7]: Note that $\forall v_1+W, v_2+W\in S, \forall a \in \mathbb{F},$ 

$a(v_1+W+(v_2+W))=a((v_1+v_2)+W)=(av_1+av_2)+W=av_1+W + (av_2+W)=a(v_1+W)+a(v_2+W).$

Therefore, VS7 is satisfied. 

~\ 

[VS8]: Note that $\forall v+W\in S, \forall a,b \in \mathbb{F},$ $$(a+b)(v+W)=(a+b)v+W=(av+bv)+W=av+W+(bv+W)=a(v+W)+b(v+W).$$

Therefore, VS8 is satisfied. 

Hence, S is proved to be a vector space with operations defined in (c). 

~\ 

\section{Section 1.4 Q10}

Note that the set of all symmetric 2 by 2 matrices is 
$S=\{\begin{pmatrix}
    a & b
    \\b & c
\end{pmatrix}: a,b,c\in \mathbb{F}\}.$ 

Note that $\forall \begin{pmatrix}
    a & b
    \\b & c
\end{pmatrix}\in S,$ $$\begin{pmatrix}
    a & b
    \\b & c
\end{pmatrix} = aM_1+bM_3+cM_2.$$ 

Hence $S \subset Span\{M_1,M_2,M_3\}.$ 

~\ 

Also note that $\forall x \in Span\{M_1,M_2,M_3\},$ $\exists a_1,a_2,a_3 \in \mathbb{F}$ such that $$a_1M_1+a_2M_2+a_3M_3=x.$$

Observe that $$x=\begin{pmatrix}
    a_1 & a_3
    \\a_3 & a_2
\end{pmatrix}.$$ 

By the definition of $S$, $x \in S.$ Therefore $ Span\{M_1,M_2,M_3\} \subset S,$ which deduces $$S=Span\{M_1,M_2,M_3\}.$$

~\ 

\section{Section 1.4 Q14}

Note that $\forall x \in Span(S_1 \cup S_2),$ $\exists a_1,a_2,...,a_m,b_1,b_2,...,b_n \in \mathbb{F}$ and $u_1,u_2,...,u_m \in S_1, v_1,v_2,...,v_m \in S_2$ such that $$x=a_1u_1+...a_mu_m+b_1v_1+...b_nv_n.$$

Note that $a_1u_1+...a_mu_m \in Span(S_1)$ and $b_1v_1+...b_nv_n \in Span(S_2).$ Which implies $$x=a_1u_1+...a_mu_m+b_1v_1+...b_nv_n \in Span(S_1)+Span(S_2).$$ 

Hence $ Span(S_1 \cup S_2) \subset Span(S_1)+Span(S_2).$ 

~\ 

Note that $\forall y \in Span(S_1)+Span(S_2)$, $\exists a_1,a_2,...,a_m,b_1,b_2,...,b_n \in \mathbb{F}$ and $u_1,u_2,...,u_m \in S_1, v_1,v_2,...,v_m \in S_2$ s.t. $$y=a_1u_1+...a_mu_m+b_1v_1+...b_nv_n.$$

That $y$ is a linear combination fo elements in $Span(S_1 \cup S_2).$ Hence $y \in Span(S_1 \cup S_2).$ 

Therefore $Span(S_1)+Span(S_2) \subset Span(S_1 \cup S_2).$ Which deduces $$Span(S_1)+Span(S_2) = Span(S_1 \cup S_2).$$

~\ 

\section{Section 1.4 Q15}

Note that $\forall x\in Span(S_1 \cap S_2),$ $\exists a_1,a_2,...,a_m \in \mathbb{F}$ and $u_1,u_2,...,u_m \in S_1 \cap S_2$ such that $$x=a_1u_1+...+a_mu_m.$$ 

Note that $u_i \in S_1, \forall i$ and $u_i \in S_2, \forall i.$ Then $x \in Span(S_1)$ and $x \in Span(S_2).$ That is $x \in Span(S_1) \cap Span(S_2).$ Which deduces that $$Span(S_1 \cap S_2) \subset Span(S_1) \cap Span(S_2).$$

~\ 

Let $S_1=\{(1,0,0),(1,1,0)\}$ and $S_2=\{(0,0,1),(0,1,1)\}.$ Note that $Span(S_1 \cap S_2)=\emptyset$ while $$Span(S_1) \cap Span(S_2)=\{(0,\lambda,0),\lambda\in \mathbb{F}\}.$$

Obviously, in this case, $Span(S_1 \cap S_2)$ and $Span(S_1) \cap Span(S_2)$ are not equal. 

~\ 

\section{Section 1.5 Q15}

We first prove the "if" part.

If $u_1=0$, note that $$1\cdot u_1+0\cdot u_2+...+0\cdot u_n=0.$$ 

Hence $S=\{u_1,u_2,...,u_n\}$ is linear dependent. 

If $u_{k+1}\in Span(\{u_1,u_2,...,u_k\})$ for some $k (1 \leq k<n).$ Then $\exists a_1,a_2,...,a_k$ such that $$u_{k+1}=a_1u_1+...+a_ku_k.$$

Hence notice that $$a_1u_1+...+a_ku_k-u_{k+1}+...+0\cdot u_n=0.$$

Hence $S=\{u_1,u_2,...,u_n\}$ is linear dependent. Thus the "if" direction is proved. 

~\

We then prove the "only if" part.

Assume $S$ is linearly dependent. Suppose $u_1\neq 0$ and $\forall 1\leq k<n,$ $$u_{k+1}\notin Span(\{u_1,...,u_k\}).$$

By the definition of $S$ being linear dependent, $\exists a_1,...,a_n$ (not all zero) such that $$0=a_1u_1+...a_nu_n.$$

Which contradicts with $u_n \notin Span(\{u_1,...,u_n-1\})$ that we assumed. 

Hence, $u_1= 0$ or $\exists 1\leq k<n,$ such that $u_{k+1}\in Span(\{u_1,...,u_k\})$ at the first place.

~\ 

\section{Section 1.5 Q18}

Suppse $S$ is linearly dependent. Then $\exists$ distinct $u_1,u_2,...,u_n \in S$ and $a_1,...,a_n \in \mathbb{F}$ (not all zero) such that $$a_1u_1+...+a_nu_n=0.$$

WLOG, let $a_1 \neq 0,$ and $u_1$ having the largest degree among those non-zero terms, denoted as $\deg(u_1).$

If $\deg(u_1)>0$, then degree of the L.H.S. is $\deg(u_1) \neq \deg(0)=0.$ Contradiction arises. 

Else if $\deg(u_1)=0,$ according to $u_1$ is having the largest degree and those polynomials do not have same degree. 

Then the L.H.S.$=a_1u_1,$ which is a nonzero constant, hence not equal to 0 on R.H.S. Contradiction arises.

Hence, S is linearly independent at the first place.

















\end{document}
