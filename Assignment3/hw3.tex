\documentclass[12pt]{article}%
\usepackage{amsfonts}
\usepackage{fancyhdr}
\usepackage{comment}
\usepackage[a4paper, top=2.5cm, bottom=2.5cm, left=2.2cm, right=2.2cm]%
{geometry}
\usepackage{times}
\usepackage{amsmath}
\usepackage{changepage}
\usepackage{stfloats}
\usepackage{amssymb}
\usepackage{graphicx}
\usepackage{indentfirst}
\setlength{\parindent}{2em}
\setcounter{MaxMatrixCols}{30}
\newtheorem{theorem}{Theorem}
\newtheorem{acknowledgement}[theorem]{Acknowledgement}
\newtheorem{algorithm}[theorem]{Algorithm}
\newtheorem{axiom}{Axiom}
\newtheorem{case}[theorem]{Case}
\newtheorem{claim}[theorem]{Claim}
\newtheorem{conclusion}[theorem]{Conclusion}
\newtheorem{condition}[theorem]{Condition}
\newtheorem{conjecture}[theorem]{Conjecture}
\newtheorem{corollary}[theorem]{Corollary}
\newtheorem{criterion}[theorem]{Criterion}
\newtheorem{definition}[theorem]{Definition}
\newtheorem{example}[theorem]{Example}
\newtheorem{exercise}[theorem]{Exercise}
\newtheorem{lemma}[theorem]{Lemma}
\newtheorem{notation}[theorem]{Notation}
\newtheorem{problem}[theorem]{Problem}
\newtheorem{proposition}[theorem]{Proposition}
\newtheorem{remark}[theorem]{Remark}
\newtheorem{solution}[theorem]{Solution}
\newtheorem{summary}[theorem]{Summary}
\newenvironment{proof}[1][Proof]{\textbf{#1.} }{\ \rule{0.5em}{0.5em}}

\usepackage{mathtools}

\newcommand{\Q}{\mathbb{Q}}
\newcommand{\R}{\mathbb{R}}
\newcommand{\C}{\mathbb{C}}
\newcommand{\Z}{\mathbb{Z}}

\begin{document}

\title{MATH2040C Homework 3}
\author{ZHENG Weijia (William, 1155124322)}
\date{\today}
\maketitle



\section{Section 2.2, Q3}

According to the question, $\beta=\{(1,0),(0,1)\}.$

Therefore, $T((1,0))=(1,1,2),T((0,1))=(-1,0,1).$ 

Then we need to find $[T((1,0))]_\gamma=[(1,1,2)]_\gamma$ and $[T((0,1))]_\gamma=[(-1,0,1)]_\gamma.$

By the question, $\gamma = \{(1,1,0),(0,1,1),(2,2,3)\}.$ 

Note that $\begin{pmatrix}1\\1\\2 \end{pmatrix}=\begin{pmatrix}1 & 0 & 2\\1 & 1& 2\\0&1&3\end{pmatrix} \begin{pmatrix} -\frac{1}{3}\\0\\\frac{2}{3} \end{pmatrix}. $

Hence $[T((1,0))]_\gamma=[(1,1,2)]_\gamma=(-\frac{1}{3},0,\frac{2}{3}).$ 

Also note that $\begin{pmatrix}-1\\0\\1 \end{pmatrix}=\begin{pmatrix}1 & 0 & 2\\1 & 1& 2\\0&1&3\end{pmatrix} \begin{pmatrix} -1\\1\\0 \end{pmatrix}. $

Therefore, $[T]_\beta^{\gamma}=\begin{pmatrix}-\frac{1}{3} & -1\\0 & 1\\\frac{2}{3} &0\end{pmatrix}.$ 

~\ 

Note that $\alpha=\{(1,2),(2,3)\}.$ And $T(1,2)=(-1,1,4)$ and $T(2,3)=(-1,2,7).$ Then we will find $[(-1,1,4)]_\gamma$ and $[-1,2,7]_\gamma.$

Note that $\begin{pmatrix}-1\\1\\4 \end{pmatrix}=\begin{pmatrix}1 & 0 & 2\\1 & 1& 2\\0&1&3\end{pmatrix} \begin{pmatrix} -\frac{7}{3}\\2\\\frac{2}{3} \end{pmatrix}.$

Hence $[T(1,2)]_\gamma = [(-1,1,4)]_\gamma = (-\frac{7}{3},2,\frac{2}{3}).$

Also note that $\begin{pmatrix}-1\\2\\7 \end{pmatrix}=\begin{pmatrix}1 & 0 & 2\\1 & 1& 2\\0&1&3\end{pmatrix} \begin{pmatrix} -\frac{11}{3}\\3\\\frac{4}{3} \end{pmatrix}.$ 

Hence $[T(2,3)]_\gamma = [(-1,2,7)]_\gamma = (-\frac{11}{3},3,\frac{4}{3}).$

Therefore, $[T]_\alpha^{\gamma}=\begin{pmatrix}-\frac{7}{3} & -\frac{11}{3} \\2 & 3\\\frac{2}{3} &\frac{4}{3}\end{pmatrix}.$ 

Done.

~\ 

\section{Section 2.2, Q5}

\subsection{(a)}

Note that $\alpha=\{\begin{pmatrix} 1& 0 \\0 & 0\end{pmatrix},\begin{pmatrix} 0& 1 \\0 & 0\end{pmatrix},\begin{pmatrix} 0 & 0 \\1 & 0\end{pmatrix},\begin{pmatrix} 0& 0 \\0 & 1\end{pmatrix}\}.$ Hence $$[T]_\alpha=[\begin{pmatrix}1&0\\0&0\end{pmatrix},\begin{pmatrix}0&0\\1&0\end{pmatrix},\begin{pmatrix}0&1\\0&0\end{pmatrix},\begin{pmatrix}0&0\\0&1\end{pmatrix}]_\alpha=\begin{pmatrix}1&0&0&0\\0&0&1&0\\0&1&0&0\\0&0&0&1\end{pmatrix}.$$

\subsection{(b)}

$[T]_\beta^{\alpha}=[T(1),T(x),T(x^2)]_{\alpha}=[\begin{pmatrix}0&2\\0&0\end{pmatrix},\begin{pmatrix}1&2\\0&0\end{pmatrix},\begin{pmatrix}0&2\\0&2\end{pmatrix}]_\alpha=\begin{pmatrix}0&1&0\\2&2&2\\0&0&0\\0&0&2\end{pmatrix}.$

\subsection{(c)}

The basis $\alpha=\{\begin{pmatrix}1&0\\0&0\end{pmatrix},\begin{pmatrix}0&1\\0&0\end{pmatrix},\begin{pmatrix}0&0\\1&0\end{pmatrix},\begin{pmatrix}0&0\\0&1\end{pmatrix}\}.$

$[T]_\alpha^{\gamma}=[tr \begin{pmatrix}1&0\\0&0\end{pmatrix},tr \begin{pmatrix}0&1\\0&0\end{pmatrix}, tr \begin{pmatrix}0&0\\1&0\end{pmatrix}, tr \begin{pmatrix}0&0\\0&1\end{pmatrix}]_\gamma=[1,1,1,1]_\gamma=\begin{pmatrix}1&1&1&1\end{pmatrix}.$

\subsection{(d)}

Recall that $\beta=\{1,x,x^2\}.$ 

$[T]_\beta^{\gamma}=[T(1),T(x),T(x^2)]_\gamma=\begin{pmatrix}1&2&4\end{pmatrix}.$

\subsection{(e)}

The basis $\alpha=\{\begin{pmatrix}1&0\\0&0\end{pmatrix},\begin{pmatrix}0&1\\0&0\end{pmatrix},\begin{pmatrix}0&0\\1&0\end{pmatrix},\begin{pmatrix}0&0\\0&1\end{pmatrix}\}.$

Because $A=\begin{pmatrix}1&-2\\0&4\end{pmatrix}$, then $$[A]_\alpha=\begin{pmatrix}1\\-2\\0\\4\end{pmatrix}.$$

\subsection{(f)}

Note that $f(x)=3-6x+x^2.$ Therefore $$[f(x)]_\beta=\begin{pmatrix}3\\-6\\1\end{pmatrix}.$$

\subsection{(g)}

$\forall a\in F,$, $[a]_\gamma=a.$

~\ 

\section{Section 2.3, Q3}

\subsection{(a)}

$[U]_\beta^{\gamma}=[U(1),U(x),U(x^2)]_\gamma=[(1,0,1),(1,0,-1),(0,1,0)]_\gamma=\begin{pmatrix}1&1&0\\0&0&1\\1&-1&0\end{pmatrix}.$

$[T]_\beta=[T(1),T(x),T(x^2)]_\beta=[2,3+3x,6x+4x^2]_\beta=\begin{pmatrix}2&3&0\\0&3&6\\0&0&4\end{pmatrix}.$

And $[UT]_\beta^{\gamma}=[UT(\beta)]_\gamma=[U(2),U(3+3x),U(6x+4x^2)]_\gamma$ 

$[UT]_\beta^{\gamma}=[(2,0,2),(6,0,0),(6,4,-6)]_\gamma=\begin{pmatrix}2&6&6\\0&0&4\\2&0&-6\end{pmatrix}.$

Verify that $$[UT]_\beta^{\gamma}=\begin{pmatrix}2&6&6\\0&0&4\\2&0&-6\end{pmatrix}=\begin{pmatrix}1&1&0\\0&0&1\\1&-1&0\end{pmatrix}\begin{pmatrix}2&3&0\\0&3&6\\0&0&4\end{pmatrix}=[U]_\beta^{\gamma}[T]_\beta.$$

Done.

\subsection{(b)}

Because $h(x)=3-2x+x^2.$ $[h(x)]_\beta=[3-2x+x^2]=\begin{pmatrix}3\\-2\\1\end{pmatrix}.$

$[U(h(x))]_\gamma=[U(3-2x+x^2)]_\gamma=[(1,1,5)]_\gamma=\begin{pmatrix}1\\1\\5\end{pmatrix}.$

Note that $$[U(h(x))]_\gamma=\begin{pmatrix}1\\1\\5\end{pmatrix}=\begin{pmatrix}1&1&0\\0&0&1\\1&-1&0\end{pmatrix}\begin{pmatrix}3\\-2\\1\end{pmatrix}=[U]_\beta^{\gamma}[h(x)]_\beta.$$

Hence the theorem 2.14 is verified.

Done.

~\ 

\section{Section 2.3, Q16}

\subsection{(a)}

Note that $T:V \to V.$ And $T^2:V \to V.$

Given that $rank(T)=rank(T^2).$ By rank-nullity theorem, $$rank(T)+nullity(T)=\dim{V}.$$

Also by rank-nullity theorem, $$rank(T^2)+nullity(T^2)=\dim{V}.$$

Hence $nullity(T^2)=nullity(T).$

It is obvious that $\forall x\in N(T),T^2(x)=T(T(x))=T(0_V)=0_V.$

Hence $N(T) \subset N(T^2).$

~\ 

Suppose that $\exists y \in N(T^2)$ such that $T(y)=0_V.$ Then $\dim{N(T^2)}>\dim{N(T)}.$

But $nullity(T)=\dim{V}-rank(T)=\dim{V}-rank(T^2)=nullity(T^2).$ Which arises contradiction.

Hence $\forall y \in N(T^2), T(y)=0_V$ at the first place, then $$N(T)=\{0_V\}.$$

Note that $0_V \in R(T),$ therefore $R(T) \cap N(T)=\{0_V\}.$ Which deduces that $$V=R(T)\bigoplus
N(T).$$

\subsection{(b)}




\end{document}
