\documentclass[12pt]{article}%
\usepackage{amsfonts}
\usepackage{fancyhdr}
\usepackage{comment}
\usepackage[a4paper, top=2.5cm, bottom=2.5cm, left=2.2cm, right=2.2cm]%
{geometry}
\usepackage{times}
\usepackage{amsmath}
\usepackage{changepage}
\usepackage{stfloats}
\usepackage{amssymb}
\usepackage{graphicx}
\usepackage{indentfirst}
\setlength{\parindent}{2em}
\setcounter{MaxMatrixCols}{30}
\newtheorem{theorem}{Theorem}
\newtheorem{acknowledgement}[theorem]{Acknowledgement}
\newtheorem{algorithm}[theorem]{Algorithm}
\newtheorem{axiom}{Axiom}
\newtheorem{case}[theorem]{Case}
\newtheorem{claim}[theorem]{Claim}
\newtheorem{conclusion}[theorem]{Conclusion}
\newtheorem{condition}[theorem]{Condition}
\newtheorem{conjecture}[theorem]{Conjecture}
\newtheorem{corollary}[theorem]{Corollary}
\newtheorem{criterion}[theorem]{Criterion}
\newtheorem{definition}[theorem]{Definition}
\newtheorem{example}[theorem]{Example}
\newtheorem{exercise}[theorem]{Exercise}
\newtheorem{lemma}[theorem]{Lemma}
\newtheorem{notation}[theorem]{Notation}
\newtheorem{problem}[theorem]{Problem}
\newtheorem{proposition}[theorem]{Proposition}
\newtheorem{remark}[theorem]{Remark}
\newtheorem{solution}[theorem]{Solution}
\newtheorem{summary}[theorem]{Summary}
\newenvironment{proof}[1][Proof]{\textbf{#1.} }{\ \rule{0.5em}{0.5em}}

\usepackage{mathtools}

\newcommand{\Q}{\mathbb{Q}}
\newcommand{\R}{\mathbb{R}}
\newcommand{\C}{\mathbb{C}}
\newcommand{\Z}{\mathbb{Z}}

\begin{document}

\title{MATH2040C Homework 5}
\author{ZHENG Weijia (William, 1155124322)}
\date{\today}
\maketitle



\section{Section 5.4, Q2(e)}

Let $w=\begin{pmatrix}2&4\\4&3\end{pmatrix}.$ Note that $w \in W$, because $w$ is symmetric.

Note that $T(w)=\begin{pmatrix}0&1\\1&0\end{pmatrix}\begin{pmatrix}2&4\\4&3\end{pmatrix}=\begin{pmatrix}4&3\\2&4\end{pmatrix},$ which is not symmetric, hence not belongs to $W$.

Therefore, by definition, $W$ is not a T-invariant subspace of $V.$

Done.


\section{Section 5.4, Q4}

$\forall g(t)$ belongs to polynomials, it can be expressed as for some $a_i,i=0,1,2,...,n$, $$g(t)=\sum_{i=0}^{n}a_i t^{i}.$$

Note that $\forall w \in W,$ we have $$g(T)(w)=\sum_{i=0}^{n}a_i T^{i}(w).$$

Because $W$ is itself a subspace, and note that $T^{i}(w) \in W, \forall i.$ Then $$\forall w \in W, g(T)(w)\in W.$$

Done.

\section{Section 5.4, Q6(d)}

Note that $z=\begin{pmatrix} 0&1\\1&0 \end{pmatrix}$, $T(z)=\begin{pmatrix} 1&1\\2&2\end{pmatrix}$ and $T^2(z)=3 \begin{pmatrix} 1&1\\2&2\end{pmatrix}.$

And hence $T^{k}(z)=3^{k-1}\begin{pmatrix} 1&1\\2&2\end{pmatrix}, \forall k\geq 1.$ 

Recall that $span\{z,T(z),T^2(z),\dots\}$ is the T-cyclic subspace of V generated by $z$.

Claim that $\{z,T(z)\}$ is a ordered basis for $span\{z,T(z),T^2(z),\dots\}$.

Note that $\forall u \in span\{z,T(z),T^2(z),\dots\},$ if $u=z$ or $u=T(z)$, then u are elements inside the basis set.

If $u=T^{k}(z)$ for some $k\geq 2,$ notice that $T^{k}(z)==3^{k-1}\begin{pmatrix} 1&1\\2&2\end{pmatrix}=3^{k-1}T(z).$

Therefore, $\{z,T(z)\}$ spans $span\{z,T(z),T^2(z),\dots\}$.

Done.

\section{Section 5.4, Q19}





\end{document}
