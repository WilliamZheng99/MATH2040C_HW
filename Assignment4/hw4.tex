\documentclass[12pt]{article}%
\usepackage{amsfonts}
\usepackage{fancyhdr}
\usepackage{comment}
\usepackage[a4paper, top=2.5cm, bottom=2.5cm, left=2.2cm, right=2.2cm]%
{geometry}
\usepackage{times}
\usepackage{amsmath}
\usepackage{changepage}
\usepackage{stfloats}
\usepackage{amssymb}
\usepackage{graphicx}
\usepackage{indentfirst}
\setlength{\parindent}{2em}
\setcounter{MaxMatrixCols}{30}
\newtheorem{theorem}{Theorem}
\newtheorem{acknowledgement}[theorem]{Acknowledgement}
\newtheorem{algorithm}[theorem]{Algorithm}
\newtheorem{axiom}{Axiom}
\newtheorem{case}[theorem]{Case}
\newtheorem{claim}[theorem]{Claim}
\newtheorem{conclusion}[theorem]{Conclusion}
\newtheorem{condition}[theorem]{Condition}
\newtheorem{conjecture}[theorem]{Conjecture}
\newtheorem{corollary}[theorem]{Corollary}
\newtheorem{criterion}[theorem]{Criterion}
\newtheorem{definition}[theorem]{Definition}
\newtheorem{example}[theorem]{Example}
\newtheorem{exercise}[theorem]{Exercise}
\newtheorem{lemma}[theorem]{Lemma}
\newtheorem{notation}[theorem]{Notation}
\newtheorem{problem}[theorem]{Problem}
\newtheorem{proposition}[theorem]{Proposition}
\newtheorem{remark}[theorem]{Remark}
\newtheorem{solution}[theorem]{Solution}
\newtheorem{summary}[theorem]{Summary}
\newenvironment{proof}[1][Proof]{\textbf{#1.} }{\ \rule{0.5em}{0.5em}}

\usepackage{mathtools}

\newcommand{\Q}{\mathbb{Q}}
\newcommand{\R}{\mathbb{R}}
\newcommand{\C}{\mathbb{C}}
\newcommand{\Z}{\mathbb{Z}}

\begin{document}

\title{MATH2040C Homework 4}
\author{ZHENG Weijia (William, 1155124322)}
\date{\today}
\maketitle



\section{Section 5.1, Q2(e)}

Given that $\beta = \{1-x+x^3,1+x^2,1,x+x^2\}.$ 

And note that $T(1-x+x^3)=-1+x-x^3.$ $T(1+x^2)=-x-x^2+x^3.$ $T(1)=x^2.$ $T(x+x^2)=-x-x^2.$

Hence $T(\beta)=\{-1+x-x^3,-x-x^2+x^3,x^2, -x-x^2\}.$

$[T]_\beta =\begin{pmatrix}-1&1&0&0\\0&-1&1&0\\0&0&-1&0\\0&0&0&-1\end{pmatrix}.$

Suppose $\beta$ is containing $T$'s eigenvectors, then $\exists \lambda \in F$ such that $$T(1+x^2)=\lambda(1+x^2).$$

Then $\lambda + \lambda x^2 = -x-x^2+x^3.$ Note that the degree of them do not equal in any sense. 

Hence $\beta$ is not a basis consisting of eigenvectors of $T.$

~\ 

\section{Section 5.1, Q2(f)}
Given that $\beta = \{\begin{pmatrix}1&0\\1&0\end{pmatrix},\begin{pmatrix}-1&2\\0&0\end{pmatrix},\begin{pmatrix}1&0\\2&0\end{pmatrix},\begin{pmatrix}-1&0\\0&2\end{pmatrix}\}.$

Note that $T\begin{pmatrix}1&0\\1&0\end{pmatrix}=\begin{pmatrix}-3&0\\-3&0\end{pmatrix}=-3\begin{pmatrix}1&0\\1&0\end{pmatrix},$

$T\begin{pmatrix}-1&2\\0&0\end{pmatrix}=\begin{pmatrix}-1&2\\0&0\end{pmatrix}=1\cdot \begin{pmatrix}-1&2\\0&0\end{pmatrix},$ 

$T\begin{pmatrix}1&0\\2&0\end{pmatrix}=\begin{pmatrix}1&0\\2&0\end{pmatrix}=1\cdot \begin{pmatrix}1&0\\2&0\end{pmatrix},$ and 

$T\begin{pmatrix}-1&0\\0&2\end{pmatrix}=\begin{pmatrix}-1&0\\0&2\end{pmatrix}=1\cdot \begin{pmatrix}-1&0\\0&2\end{pmatrix}.$

Hence we deduce that $\beta$ is a basis consisting of eigenvectors of $T$.

~\ 

\section{Section 5.1, Q3(d)}
\subsection{(i)}
Given that $A = \begin{pmatrix}2&0&-1\\4&1&-4\\2&0&-1\end{pmatrix}$, then its characteristic polynomial is $$f_{A}(t)=\det{\begin{pmatrix}2-t&0&-1\\4&1-t&-4\\2&0&-1-t\end{pmatrix}}=-t(t-1)^2.$$

Observe the $f_{A}(t)$'s zeros, we have A should have 2 eigenvalues: 1 and 0.

\subsection{(ii)}
For eigenvalue 1, the corresponding eigenvectors should be in the span of set $$\{\begin{pmatrix}1\\0\\1\end{pmatrix},\begin{pmatrix}0\\1\\0\end{pmatrix}\}.$$

For eigenvalue 0, the corresponding eigenvectors should be in the span of set $$\{\begin{pmatrix}1\\4\\2\end{pmatrix}\}.$$

\subsection{(iii)}
In this case, the $n=3, F = \mathbb{R}.$ So $F^3 = \mathbb{R}^3.$

Note that $\{ \begin{pmatrix}1\\0\\1\end{pmatrix}, \begin{pmatrix}0\\1\\0\end{pmatrix}, \begin{pmatrix}1\\4\\2\end{pmatrix}     \}$ is a 3-linear-indenpendent set. Hence it is a basis of $\mathbb{R}^3.$ And by our conclusion above, these 3 vectors are eigenvectors of $A$.

\subsection{(iv)}
Let $Q=\begin{pmatrix}1&0&1\\0&1&4\\1&0&2\end{pmatrix}$. Note that $Q$ is invertible and $Q^{-1}=\begin{pmatrix}2&0&-1\\4&1&-4\\-1&0&1\end{pmatrix}.$ 

Note that $$Q^{-1}AQ =\begin{pmatrix}1&0&0\\0&1&0\\0&0&0\end{pmatrix}.$$ 

~\ 

\section{Section 5.1, Q4(h)}
Let $\beta$ be the standard basis.
Note that $[T]_{\beta}= \begin{pmatrix}0&0&0&1\\0&1&0&0\\0&0&1&0\\1&0&0&0\end{pmatrix}.$ By extracting its characteristic polynomial, it is $$f_{T}(t)=(t-1)^3(t+1)=0.$$

And note that their corresponding eigenvectors to be $\{\begin{pmatrix}0\\1\\0\\0\end{pmatrix},\begin{pmatrix}0\\0\\1\\0\end{pmatrix},\begin{pmatrix}1\\0\\0\\1\end{pmatrix},\begin{pmatrix}1\\0\\0\\-1\end{pmatrix}\}.$

Note that by the diagnoalizability of $[T]_\beta$, (for its every eigenvalue: 1 and -1: algebraic multiplicity equals geometric multiplicity) we have $$\begin{pmatrix}1&0&0&0\\0&1&0&0\\0&0&1&0\\0&0&0&-1\end{pmatrix}=Q^{-1}[T]_\beta Q.$$

Where $Q=\begin{pmatrix}0&0&-1&1\\-1&0&0&0\\0&1&0&0\\0&0&-1&-1\end{pmatrix}.$

Regard Q as a change of basis matrix from another basis $\gamma$ to our known standard basis $\beta.$ Therefore, $Q=[I]_{\gamma}^{\beta}.$

Let $\gamma=\{y_1,y_2,y_3,y_4\}.$ Therefore $$[y_1,y_2,y_3,y_4]_\beta =\begin{pmatrix}0&0&-1&1\\-1&0&0&0\\0&1&0&0\\0&0&-1&-1\end{pmatrix}. $$

Hence, $y_1 = \begin{pmatrix} 0&-1\\0&0\end{pmatrix}$, $y_2 = \begin{pmatrix}0&0\\1&0\end{pmatrix}$, $y_3 = \begin{pmatrix} -1&0\\0&-1\end{pmatrix}$, $y_4 = \begin{pmatrix} 1&0\\0&-1\end{pmatrix}.$

Note that $[T]_\gamma=\begin{pmatrix}0&0&-1&1\\-1&0&0&0\\0&1&0&0\\0&0&-1&-1\end{pmatrix}.$ So $\gamma$ is the ordered basis we need to find.

~\ 

\section{Section 5.1, Q4(e)}

Let $\beta = \{1+x, -3-13x+4x^2, -3+x\}$ be a ordered basis. Then $$[T]_\beta=[4x+4,8x^2-26x-6,0]_\beta=\begin{pmatrix}4&0&0\\0&2&0\\0&0&0\end{pmatrix}.$$

Which is a diagnoal matrix. Hence the $\beta$ is what we want to find. And the eigenvalues of $T$ are 4,2 and 0, with corresponding eigenvectors elements of the ordered basis $\beta.$

~\ 

\section{Section 5.1, Q17}




~\ 

\section{Section 5.1, Q18}
\subsection{(a)}
If $A$ is not invertible, then let $c=0.$ We have $$\det{(A+cB)}=\det{A}=0.$$ Since $A$ is singular as we supposed.

If $A$ is invertible, then note that $A = AB^{-1}B$, then $$\det{(A+cB)}=\det{AB^{-1}B+cB}=\det{(AB^{-1}+cI)}\det{(B)}.$$

Note that $\det{(B)}\neq 0$ and $\det{(AB^{-1}+cI)}=0$ if $-c$ is the eigenvalue of $AB^{-1}$. And by the fundemantal theorem of algebra, there must exist such c.

Done.

\subsection{(b)}
Let $A=\begin{pmatrix}2&1\\0&1\end{pmatrix}$ and $B=\begin{pmatrix}0&1\\0&0\end{pmatrix}.$

Note that $\det{(A)}=2\neq 0.$ And $\forall c\in \mathbb{C},$ $$\det{(A+cB)}=\det{(\begin{pmatrix}2&1+c\\0&1\end{pmatrix})}=2\neq 0.$$

Therefore, A and A+cB are both invertible. 
~\ 

\section{Section 5.2, Q3(c)}

~\ 

\section{Section 5.2, Q8}

~\ 

\section{Section 5.2, Q13}

~\ 

\end{document}
